\documentclass{article} % For LaTeX2e
\usepackage{nips10submit_e,times}
%\usepackage{xcolor}
\usepackage{colortbl}
%\documentstyle[nips10submit_09,times,art10]{article} % For LaTeX 2.09


\title{Predicting Course Grades}


\author{
Martin Hunt, Sharon Lin, Chinmay Kulkarni \\
\texttt{ \{chinmay, mghunt, sharonl\}@stanford.edu}
}

% The \author macro works with any number of authors. There are two commands
% used to separate the names and addresses of multiple authors: \And and \AND.
%
% Using \And between authors leaves it to \LaTeX{} to determine where to break
% the lines. Using \AND forces a linebreak at that point. So, if \LaTeX{}
% puts 3 of 4 authors names on the first line, and the last on the second
% line, try using \AND instead of \And before the third author name.

\newcommand{\fix}{\marginpar{FIX}}
\newcommand{\new}{\marginpar{NEW}}

\nipsfinalcopy % Uncomment for camera-ready version

\begin{document}


\maketitle

\begin{abstract}
Recommendation systems have been used in a variety of domains, ranging from online-joke recommendations to automatic educational course recommendations.
To better evaluate recommendations, the user can be presented with estimations for the various attributes of a recommendation. We attempt to construct one particular potential value-predictor: the estimated grade a student will receive for a given course. Data is from the CourseRank recommendation system.
\end{abstract}

\section{Introduction}
\label{sec:intro}
Recommendation systems attack the general problem of recommending users items that are likely to be of interest to them. This is typically viewed as a problem of predicting the ``rating'' of a given item, based on known ratings of other items, followed by finding items that have the highest ratings. Recommendation systems have been used in a variety of applications, and depending on the application, the \textit{items} could be physical goods, movies, research papers, webpages, educational courses, etc~\cite{schafer1999recommender}. Similarly, the \textit{ratings} could either be explicitly provided by the user (e.g. using a 1-to-5 star rating, or a Likert scale), or obtained implicitly by the system (e.g. the number of times a webpage was visited).

Recommendation systems are primarily either model-based or memory-based. A model-based system abstracts a model that predicts ratings for a given user and item. A memory-based system uses unsupervised learning to cluster user-item data (either the users or the items may be clustered) based on the known ratings. An unknown rating is then approximated with a weighted average of known ratings, the weight being the distance of the unrated item and the rated items. Memory-based systems have been particularly popular, since they makes fewer assumptions regarding what factors contributed to a rating, and are more likely to produce diverse recommendations than a model-based approach. 

While recommendation systems recommend a set of items to a user, the user still has to evaluate these recommendations in order to find the most useful ones. In this paper, we explore how users can be assisted in this task. One particular aspect that we focus on is value prediction for item-attributes. This is useful for answering questions such as, ``how \textit{suspenseful} is this recommended movie?'', or ``how \textit{important} is this research paper?''. 

We think value prediction for item-attributes is especially useful for a) attributes whose values vary by user (e.g. ``interestingness'' for a book) b) attribute values that can only be measured after the item is consumed (e.g. ``durability'' for furniture) and c) attributes that may change based on whether the user has consumed other items (e.g. ``understandability'' for movie sequels [TODO less cheesy examples?]) 

In our system, the exemplar question we try to answer is ``what \textit{grade} will I make?'' for a University course recommendation. Course grades (fortunately) have all the above characteristics: the grade varies by user, is only available after the course is taken and potentially depends on other courses the student has taken in the past.

\section{Prior work}
Memory-based recommendation systems make the assumption that users who have rated items similarly (i.e. favorably or unfavorably) in the past will also rate a new item similarly, so these systems typically expose, in addition to the recommended item, either other similar items the user has rated favorably in the past, or other users with tastes similar to the current user who rated the recommended item favorably. This information helps users trust recommendations~\cite{o2005trust, adomavicius2005toward}, we argue that it still does not help them directly evaluate them for two reasons.

First, exposing similar items often provides no information about attribute values-- for instance, if a user is buying furniture and has rated furniture that was durable favorably in the past, the system cannot predict durability, despite showing that the current recommendation is ``similar'' to his past purchases. 

Second, even if information is provided, human perception of similarity is non-metric, so humans still can't accurately estimate attributes values. In particular, similarity perception is not symmetric (for instance, China is perceived to be more similar to North Korea than North Korea is to China), and it does not follow the triangle-inequality (similarity of (A with B) + (B with C) could be less than A with C)~\cite{tversky1982similarity}.

Model-based approaches have problems with evaluation too. Models are often based on Bayesian networks, LSA or MDPs-- while these models have good accuracy, they can't predict attributes they weren't trained for [TODO can someone rephrase?].

\section{Predicting grades}
Our data is drawn from the CourseRank website, which is a course-recommendation website run at Stanford.


\subsection{SVM}
We trained an SVM to detect A versus non-A grades, where doing well in a course meant getting an A grade.  We also trained a multi-class SVM to predict grades across the whole letter grade spectrum from A+ to F. One SVM was trained per course.

All SVMs were  trained using a Gaussian kernel. The C and gamma parameters for the kernel were chosen via a rough grid-based search on C={128 64 32 8 4 2 1} and gamma={1/numFeatures 2/numFeatures 4/numFeatures 8/numFeatures} to maximize the 10-fold cross validation accuracy. The multi-class SVM was implemented using multiple SVMs that compared grades higher than a certain threshold against the grades lower than the threshold. That is, there is one SVM that predicts each of  A+ vs other grades, A or above vs. below A, A- or above vs. below A-, and so on for each unique grade appearing in the course. The final predicted grade is then decided by a majority vote among all the individual SVMs.


Given the CourseRank transcript data, we chose six features that we hypothesized may affect course grades-- the students' previous course grades, recent GPA by department (last three quarters), major, concurrent courses, planned weekly workload, and the number of courses previously taken.

These features provide some information on the students' experience with related courses and subjects at Stanford, whether the course is a requirement or an elective for their major, and how much time they have to spend on the course. The previous course grades, recent GPA by department, student major, and concurrent courses were represented by vectors where each index corresponded to one course, department, or major.






\subsection{Collaborative filtering}


%Need to add CF stuff to results and conclusions

\section{Results}

\subsection{SVM Results}

We trained A vs. non-A and multi-class SVMs for each of 17 courses that had at least 100 student records. We compared the 10-fold cross-validation accuracy against the average course grade and student's GPA baselines. For the student's GPA baseline, if a student has not taken a course before, his or her grade is predicted to be the average course grade. The results are shown in the Tables \ref{svm-accuracy-table} and \ref{svm-mae-table} below.

\subsubsection{Prediction Accuracy Compared to Baselines}

On average, both types of SVMs outperformed the mean course grade baseline and are comparable to the student's GPA baseline. Overall, the multi-class SVM correctly predicts a slightly greater percentage of students within a half letter grade when compared to the baselines.

\begin{table}[htbp]\scriptsize
\label{svm-accuracy-table}
\begin{center}
\begin{tabular}{lr}
\multicolumn{1}{c}{\bf Predictor}  &\multicolumn{1}{c}{\bf Average Accuracy}
\\ \hline \\
A vs. non-A SVM &64.2\% \\
Student's GPA   &66.4\% \\
Mean course grade &60.6\% \\
\end{tabular}
\caption{Average accuracy for the A vs. non-A SVMs compared to the baselines (higher is better)}
\end{center}
\end{table}


\begin{table}[htbp]\scriptsize
\label{svm-mae-table}
\begin{center}
\begin{tabular}{lrr}
\multicolumn{1}{c}{\bf Predictor}  &\multicolumn{1}{c}{\bf Average MAE}  &\multicolumn{1}{c}{\bf Within a half letter grade}
\\ \hline \\
Multi-class SVM &1.38 & 65.2\%\\
Student's GPA   &1.37 & 61.1\%\\
Mean course grade &1.45 &57.0\% \\
\end{tabular}
\caption{Comparing the multi-class SVMs compared to the baselines. An error of 1 corresponds to a half letter grade error, while an error of 2 corresponds to a full letter grade error.}
\end{center}
\end{table}


\subsubsection{Most Important Features}

We evaluated the contribution of each set of features by training the A vs. non-A SVM exclusively on each set  and comparing the 10-fold cross-validation accuracies against the baseline statistical accuracy. Table \ref{svm-features-table} shows the results.

Overall, the top four features in order were previous course history, recent grades by department, student's major, and concurrent courses. Weekly workload did not contribute to the accuracy of any of the selected courses, and the number of taken courses only influenced accuracy in CS161 out of the 17 courses tested.

However, across different courses, the features that most influenced prediction accuracy varied. For example, in CS161, all of the features except for weekly workload contributed to the prediction rate, and the student's previous course grades were the most significant predictor. On the other hand, in ARTSTUDI60, the student's major was the most significant predictor. In CHEM31A, none of the features contributed to the accuracy, suggesting that different factors may better characterize course grades in this case.

\begin{table}[t]\scriptsize
\label{svm-features-table}
\begin{center}
\begin{tabular}{lrrrrrr}
\multicolumn{1}{c}{\bf Course}  &\multicolumn{1}{c}{\bf Dept. GPA} &\multicolumn{1}{c}{\bf Concurr. courses} &\multicolumn{1}{c}{\bf Course grades} &\multicolumn{1}{c}{\bf Major} &\multicolumn{1}{c}{\bf Workload} &\multicolumn{1}{c}{\bf Num. Prev. Courses}
\\ \hline \\
ARTSTUDI60&	0&	1.83&	1.83& \cellcolor[gray]{0.8}5.50&	0&	0\\
CHEM31A	& 0	& 0 &	0 &	0 &	0&	0\\
CHEM33&\cellcolor[gray]{0.8}	3.11&	0.35&	1.29&	0.09&	0&	0\\
CS105&	0.17&	0.50&\cellcolor[gray]{0.8}2.67&	0.50&	0&	0\\
CS106A&	0.63&	0.00&	1.40&\cellcolor[gray]{0.8}	3.29&	0&	0\\
CS161	&10.21&	8.80&\cellcolor[gray]{0.8}16.90&	5.99&	0&	11.27\\
CS229	&0&\cellcolor[gray]{0.8}	1.96&	0&	0&	0&	0\\
EE108B&	0&\cellcolor[gray]{0.8}	1.86&	0&	1.24&	0&	0\\
HUMBIO2A&	0.31&	0&\cellcolor[gray]{0.8}	6.16&	0&	0&	0\\
HUMBIO2B&	5.24&	0&\cellcolor[gray]{0.8}	7.26&	0&	0&	0\\
IHUM2&\cellcolor[gray]{0.8}	7.99&	1.91&	6.25&	5.90&	0&	0\\
IHUM57&	0&\cellcolor[gray]{0.8}	1.01&	0&	0&	0&	0\\
IHUM63&	0&\cellcolor[gray]{0.8}	0.66&	0&	0&	0&	0\\
MATH42&\cellcolor[gray]{0.8}	4.40&	0&	0.51&	0&	0	&0\\
PHYSICS43&	4.93&	1.15&\cellcolor[gray]{0.8}	5.92&	1.15&	0&	0\\
PSYCH1&\cellcolor[gray]{0.8}	1.67&	1.12&	1.12&	1.49&	0&	0\\
POLISCI1&	0.34&	1.02&\cellcolor[gray]{0.8}	3.07&	2.39&	0&	0\\
\\ \hline \\
Average &2.29&	1.30&	3.20&	1.62&	0.00&	0.66\\

\end{tabular}
\caption{Percent accuracy increases for A vs. non-A SVM when including only one of the feature types. The largest contributor for each course is highlighted.}
\end{center}
\end{table}

\subsection{Results of Collaborative Filtering}

For CF, we filtered data to remove all courses with fewer than 3 grades and all students with fewer than 4 courses, resulting in 5,930 students and 3,603 courses.  We tested against the entire dataset and specific classes.  For the entire dataset, $10\%$ of the students were randomly separated with 3 courses per student removed.  Each student was compared with the remaining $90\%$ (5337) to compute the top-$n$ most similar students. The results are shown in Table~\ref{cf-mae-table}.

The primary reason the estimates from CF are not much better than the estimate based on students' GPAs is that the dataset is too sparse to find similar users.  Upon analysis of similar users, we find that the similarity coefficients are very small ($O(10^-3)$).  We further validate this by noting that analysis shows the error of the student GPA estimates and the CF estimates are highly correlated.  Because the similarity coefficients determine the deviation of the estimate from the average, the CF estimates generated with small similarity coefficients are very similar to the student GPA estimates.

Recognizing that this user-based CF approach requires more data to be effective, we experimented with tuning various CF parameters in a non-rigorous manner with minor reductions in the MAE ($O(0.1)$).  We tried various values of $n$ from all students to $5$ at logarithmic intervals and found that smaller $n$ tended to reduce the MAE.  We also looked at the similarity coefficients: as defined in~\cite{breese}, the similarities are normalized before computing the weighted average.  By increasing the normalization coefficients, we further reduced the MAE.  We hypothesize that increasing the weights favors outliers and skews the estimate away from the average.  We also tried removing the influence of negatively correlated students as both Herlocker and Yao indicate that this improves error.  Though all these adjustments reduced the MAE, none significantly improved the estimates. 

For comparison with support vector machines, we ran tests on three random courses to compute the MAE for just those courses.  Here, for the chosen course $c$, we consider all students $S$ with $c \in C_{S_i} \forall{i}$.  We compare each student $S_i$ with students $S_j, j \neq i$ by computing the similarity $\textrm{sim}(C_{S_J}, C_{S_I} \\ c)$ and then using the weighted average as before to estimate $S_i$'s grade in $c$.  We used courses with over 50 students to run these tests.  The MAE for these tests did not differ significantly from the average student estimates.

\begin{table}[t]\scriptsize
\label{cf-mae-table}
\begin{center}

\begin{tabular}{lrrrr}
    
    &\multicolumn{1}{c}{\bf cosine sim}  &\multicolumn{1}{c}{\bf pearson's corr} &\multicolumn{1}{c}{\bf student's gpa} &\multicolumn{1}{c}{\bf mean course grade}
    \\ \hline \\
    MAE & 1.3896    &1.3704    &1.4231    &3.2140 \\
    $L_2$ error & 3.9439    &3.7505    &4.0451   &14.4421
\end{tabular}
\caption{MAE and $L_2$ error for estimates using collaborative filtering averaged over 25 test runs.  Each test estimated approximately 1,500 grades using 10,453 known course grades to compute the similarities using a base corpus of 5337 students.}

\end{center}
\end{table}

\section{Conclusion and Future Work}

Past course history can only go so far in predicting future course grades. A significant percentage of students in many of the courses had not taken any courses previously, which may partially explain the low accuracy of the baselines and the SVMs. In addition, there may be more promising features that characterize the current quarter if course schedule data were available, such as potential schedule and deadline conflicts between concurrent courses and student activities. Although the concurrent courses features was intended to help capture some of the effects of schedule conflict, it is flawed in that most courses do not have a fixed schedule and instead change term by term. Thus courses that have conflicted in the past may not conflict in the future.

It is also possible that the unexpected occurrences throughout the quarter may have greater influence on grades. For example, a student may just not have studied enough for a final exam, which heavily influences his or her grade. Individual personality factors may also play a part, where doing poorly on a similar course in the past may drive some students to work harder on a similar course in the future.

Improving the collaborative filtering results would most likely require additional data from the CourseRank system.  We would hope to find more similarities between students on which to base our estimations.  This may not be possible: though more grades will be added to the system each year, the relevancy of the older grades will diminish.

An alternative approach to address data sparsity might be to broaden our initial hypothesis.  We could broaden our hypothesis that similar students will perform similarly in courses.  Instead, given a student $s$ and course $c$, we could try to find students that have not only taken the same courses as $s$ an also taken $c$, but who have taken similar courses to both $c$ and courses similar to those taken by $s$.  Jun Wang, et. all have demonstrated a method similar to this that combines user-based and item-based collaborative filtering to overcome sparseness \cite{fusion}.  

Finally, it is possible that we see poor performance because the underlying assumption that similar students in courses from some set $C_a$ will receive similar grades in some other set $C_b$ may be false.  Factors such as concurrent course workload, interest in course, different professors, and others may instead be responsible for deviations in expected course grades.  Some of these external factors may be irrelevant as our analysis of SVMs indicates, but others may require collection of additional data before we can significantly improve course grade estimates.





%sample is the nips sample formatting guidlines, comment out when not needed anymore

%\input{sample}
\bibliographystyle{abbrv}
\bibliography{refs}
\end{document}
