\documentclass{article} % For LaTeX2e
\usepackage{nips10submit_e,times}
%\usepackage{xcolor}
\usepackage{url}
\usepackage{colortbl}
%\documentstyle[nips10submit_09,times,art10]{article} % For LaTeX 2.09


\title{Predicting Course Grades}


\author{
Martin Hunt, Sharon Lin, Chinmay Kulkarni \\
\texttt{ \{chinmay, mghunt, sharonl\}@stanford.edu}
}

% The \author macro works with any number of authors. There are two commands
% used to separate the names and addresses of multiple authors: \And and \AND.
%
% Using \And between authors leaves it to \LaTeX{} to determine where to break
% the lines. Using \AND forces a linebreak at that point. So, if \LaTeX{}
% puts 3 of 4 authors names on the first line, and the last on the second
% line, try using \AND instead of \And before the third author name.

\newcommand{\fix}{\marginpar{FIX}}
\newcommand{\new}{\marginpar{NEW}}

\nipsfinalcopy % Uncomment for camera-ready version

\begin{document}


\maketitle

\begin{abstract}
Recommendation systems have been used in a variety of domains, ranging from online-joke recommendations to automatic educational course recommendations (CourseRank).
To assess the value of the recommendations, users can be presented with estimations of various attributes of a recommendation, helping each user select the recommendation that aligns best with his or her goals and values.  Using the data from the CourseRank recommendation system, we attempt to construct one potential value-predictor: the estimated grade a student will receive for a given course.  Both SVMs and collaborative filtering techniques performed well, but neither could provide significantly better grade predictions than a baseline estimate of the average grade for each student.
\end{abstract}

\section{Introduction}
\label{sec:intro}
Recommendation systems attack the general problem of recommending users items that are likely to be of interest to them. This is typically viewed as a problem of predicting the ``rating'' of a given item, based on known ratings of other items, followed by finding items that have the highest ratings. Recommendation systems have been used in a variety of applications, and depending on the application, the \textit{items} could be physical goods, movies, research papers, webpages, educational courses, etc~\cite{schafer1999recommender}. Similarly, the \textit{ratings} could either be explicitly provided by the user (e.g. using a 1-to-5 star rating, or a Likert scale), or obtained implicitly by the system (e.g. the number of times a webpage was visited).

Recommendation systems are primarily either model-based or memory-based. A model-based system abstracts a model that predicts ratings for a given user and item. A memory-based system uses unsupervised learning to cluster user-item data (either the users or the items may be clustered) based on the known ratings. An unknown rating is then approximated with a weighted average of known ratings, the weight being the distance of the unrated item and the rated items. Memory-based systems have been particularly popular, since they makes fewer assumptions regarding what factors contributed to a rating, and are more likely to produce diverse recommendations than a model-based approach. 

While recommendation systems recommend a set of items to a user, the user still has to evaluate these recommendations in order to find the most useful ones. In this paper, we explore how users can be assisted in this task. One particular aspect that we focus on is value prediction for item-attributes. This is useful for answering questions such as, ``how \textit{suspenseful} is this recommended movie?'', or ``how \textit{important} is this research paper?''. 

We think value prediction for item-attributes is especially useful for a) attributes whose values vary by user (e.g. ``interestingness'' for a book) b) attribute values that can only be measured after the item is consumed (e.g. ``durability'' for furniture) and c) attributes that may change based on whether the user has consumed other items (e.g. ``understandability'' for movie sequels [TODO less cheesy examples?]) 

In our system, the exemplar question we try to answer is ``what \textit{grade} will I make?'' for a University course recommendation. Course grades (fortunately) have all the above characteristics: the grade varies by user, is only available after the course is taken and potentially depends on other courses the student has taken in the past.

\section{Prior Work}
Memory-based recommendation systems make the assumption that users who have rated items similarly (i.e. favorably or unfavorably) in the past will also rate a new item similarly, so these systems typically expose, in addition to the recommended item, either other similar items the user has rated favorably in the past, or other users with tastes similar to the current user who rated the recommended item favorably. This information helps users trust recommendations~\cite{o2005trust, adomavicius2005toward}, we argue that it still does not help them directly evaluate them for two reasons.

First, exposing similar items often provides no information about attribute values-- for instance, if a user is buying furniture and has rated furniture that was durable favorably in the past, the system cannot predict durability, despite showing that the current recommendation is ``similar'' to his past purchases. 

Second, even if information is provided, human perception of similarity is non-metric, so humans still can't accurately estimate attributes values. In particular, similarity perception is not symmetric (for instance, China is perceived to be more similar to North Korea than North Korea is to China), and it does not follow the triangle-inequality (similarity of (A with B) + (B with C) could be less than A with C)~\cite{tversky1982similarity}.

Model-based approaches have problems with evaluation too. Models are often based on Bayesian networks, LSA or MDPs-- while these models have good accuracy, they can't predict attributes they weren't trained for [TODO can someone rephrase?].

\section{Predicting Grades}
Our data is drawn from the CourseRank website, which is a course-recommendation website run at Stanford.

\subsection{SVM}
we trained an SVM to detect A versus non-A grades, where doing well in a course meant getting an A grade.  We also trained a multi-class SVM to predict grades across the whole letter grade spectrum from A+ to F. One SVM was trained per course.

All SVMs were  trained using a Gaussian kernel. The C and gamma parameters for the kernel were chosen via a rough grid-based search on C=[128 64 32 8 4 2 1] and gamma=[1/numFeatures 2/numFeatures 4/numFeatures 8/numFeatures] to maximize the 10-fold cross validation accuracy. The multi-class SVM was implemented using multiple SVMs that compared grades higher than a certain threshold against the grades lower than the threshold. That is, there is one SVM that predicts each of  A+ vs other grades, A or above vs. below A, A- or above vs. below A-, and so on for each unique grade appearing in the course. The final predicted grade is then decided by a majority vote among all the individual SVMs.

Features
Given the CourseRank transcript data, we chose six features that we hypothesized may affect course grades.
\begin{itemize}
\item Previous course grades 
\item Recent GPA by department (last three quarters)
\item Student�s major
\item Concurrent courses the student is taking
\item Weekly workload of quarter
\item Number of courses previously taken
\end{itemize}

These features provide some information on the students� experience with related courses and subjects at Stanford, whether the course is a requirement or an elective for their major, and how much time they have to spend on the course.


\subsection{Collaborative Filtering}

We consider another model for estimating grades based on the same methods used in the original recommendation system.  Here, we make the hypothesis that similar students will make similar grades.  Thus, for a student $s$ with course history $C_s$ and course $c \notin C_s$, we find students $s_i \in S$ with $|C_{s_i} \cup C_s| > 0$ and $c \in C_{s_i}$ (students who have taken some of the same courses as $s$ as well as course $c$).  Then $u$'s grade in $c$ can be estimated as a weighted average of the grades $g_{s_i}$.  

To determine the relative weights of users, we used cosine similarity and Pearson's correlation as the primary two measures.  Other measures of similarity such as Kendall, Spearman, and adjusted cosine similarity have been explored in~\cite{herlocker} and~\cite{yu}, but they tend to perform similarly or worse.  We expect our choice of similarity will have little impact on the predicted grades.

For two student (sparse) vectors $S_a$ and $S_b$ with each element corresponding to the student's grade in a course, we compute the similarity using:

\begin{center}
\begin{tabular}{cc}
$\textrm{similarity} = \frac{S_a \cdot S_b}{\|S_a\|\|S_b\|}$
&
$\textrm{Pearson's} = \frac{\textrm{cov}(S_a, S_b)}{\sigma_{S_a}\sigma_{S_b}} = \frac{\sum ^n _{i=1}(S_{a}^{(i)} - \overline{S}_a)(S_{b}^{(i)} - \overline{S}_b)}{\sqrt{\sum ^n _{i=1}(S_a^{(i)} - \overline{S}_a)^2} \sqrt{\sum ^n _{i=1}(S_b^{(i)} - \overline{S}_b)^2}}$.
\end{tabular}
\end{center}

Grade predictions for CF follow the multiclass SVM model with 1: A+, 2: A, 3: A-, $\ldots$, 13: F.  Other implementation details follow the formulas and algorithms outlined by~\cite{breese} and~\cite{cftoolkit}.



%Need to add CF stuff to results and conclusions

\section{Results}

\subsection{SVM Results}

We trained A vs. non-A and multi-class SVMs for each of 17 courses that had at least 100 student records. We compared the 10-fold cross-validation accuracy against the average course grade and student's GPA baselines. For the student's GPA baseline, if a student has not taken a course before, his or her grade is predicted to be the average course grade. The results are shown in the Tables \ref{svm-accuracy-table} and \ref{svm-mae-table} below.

\subsubsection{Prediction Accuracy Compared to Baselines}

On average, both types of SVMs outperformed the mean course grade baseline and are comparable to the student's GPA baseline. Overall, the multi-class SVM correctly predicts a slightly greater percentage of students within a half letter grade when compared to the baselines.

\begin{table}[htbp]\scriptsize
\label{svm-accuracy-table}
\begin{center}
\begin{tabular}{lr}
\multicolumn{1}{c}{\bf Predictor}  &\multicolumn{1}{c}{\bf Average Accuracy}
\\ \hline \\
A vs. non-A SVM &64.2\% \\
Student's GPA   &66.4\% \\
Mean course grade &60.6\% \\
\end{tabular}
\caption{Average accuracy for the A vs. non-A SVMs compared to the baselines (higher is better)}
\end{center}
\end{table}


\begin{table}[htbp]\scriptsize
\label{svm-mae-table}
\begin{center}
\begin{tabular}{lrr}
\multicolumn{1}{c}{\bf Predictor}  &\multicolumn{1}{c}{\bf Average MAE}  &\multicolumn{1}{c}{\bf Within a half letter grade}
\\ \hline \\
Multi-class SVM &1.38 & 65.2\%\\
Student's GPA   &1.37 & 61.1\%\\
Mean course grade &1.45 &57.0\% \\
\end{tabular}
\caption{Comparing the multi-class SVMs compared to the baselines. An error of 1 corresponds to a half letter grade error, while an error of 2 corresponds to a full letter grade error.}
\end{center}
\end{table}


\subsubsection{Most Important Features}

We evaluated the contribution of each set of features by training the A vs. non-A SVM exclusively on each set  and comparing the 10-fold cross-validation accuracies against the baseline statistical accuracy. Table \ref{svm-features-table} shows the results.

Overall, the top four features in order were previous course history, recent grades by department, student's major, and concurrent courses. Weekly workload did not contribute to the accuracy of any of the selected courses, and the number of taken courses only influenced accuracy in CS161 out of the 17 courses tested.

However, across different courses, the features that most influenced prediction accuracy varied. For example, in CS161, all of the features except for weekly workload contributed to the prediction rate, and the student's previous course grades were the most significant predictor. On the other hand, in ARTSTUDI60, the student's major was the most significant predictor. In CHEM31A, none of the features contributed to the accuracy, suggesting that different factors may better characterize course grades in this case.

\begin{table}[t]\scriptsize
\label{svm-features-table}
\begin{center}
\begin{tabular}{lrrrrrr}
\multicolumn{1}{c}{\bf Course}  &\multicolumn{1}{c}{\bf Dept. GPA} &\multicolumn{1}{c}{\bf Concurr. courses} &\multicolumn{1}{c}{\bf Course grades} &\multicolumn{1}{c}{\bf Major} &\multicolumn{1}{c}{\bf Workload} &\multicolumn{1}{c}{\bf Num. Prev. Courses}
\\ \hline \\
ARTSTUDI60&	0&	1.83&	1.83& \cellcolor[gray]{0.8}5.50&	0&	0\\
CHEM31A	& 0	& 0 &	0 &	0 &	0&	0\\
CHEM33&\cellcolor[gray]{0.8}	3.11&	0.35&	1.29&	0.09&	0&	0\\
CS105&	0.17&	0.50&\cellcolor[gray]{0.8}2.67&	0.50&	0&	0\\
CS106A&	0.63&	0.00&	1.40&\cellcolor[gray]{0.8}	3.29&	0&	0\\
CS161	&10.21&	8.80&\cellcolor[gray]{0.8}16.90&	5.99&	0&	11.27\\
CS229	&0&\cellcolor[gray]{0.8}	1.96&	0&	0&	0&	0\\
EE108B&	0&\cellcolor[gray]{0.8}	1.86&	0&	1.24&	0&	0\\
HUMBIO2A&	0.31&	0&\cellcolor[gray]{0.8}	6.16&	0&	0&	0\\
HUMBIO2B&	5.24&	0&\cellcolor[gray]{0.8}	7.26&	0&	0&	0\\
IHUM2&\cellcolor[gray]{0.8}	7.99&	1.91&	6.25&	5.90&	0&	0\\
IHUM57&	0&\cellcolor[gray]{0.8}	1.01&	0&	0&	0&	0\\
IHUM63&	0&\cellcolor[gray]{0.8}	0.66&	0&	0&	0&	0\\
MATH42&\cellcolor[gray]{0.8}	4.40&	0&	0.51&	0&	0	&0\\
PHYSICS43&	4.93&	1.15&\cellcolor[gray]{0.8}	5.92&	1.15&	0&	0\\
PSYCH1&\cellcolor[gray]{0.8}	1.67&	1.12&	1.12&	1.49&	0&	0\\
POLISCI1&	0.34&	1.02&\cellcolor[gray]{0.8}	3.07&	2.39&	0&	0\\
\\ \hline \\
Average &2.29&	1.30&	3.20&	1.62&	0.00&	0.66\\

\end{tabular}
\caption{Percent accuracy increases for A vs. non-A SVM when including only one of the feature types. The largest contributor for each course is highlighted.}
\end{center}
\end{table}

\subsection{Results of Collaborative Filtering}

For CF, we filtered data to remove all courses with fewer than 3 grades and all students with fewer than 4 courses, resulting in 5,930 students and 3,603 courses.  We tested against the entire dataset and specific classes.  For the entire dataset, $10\%$ of the students were randomly separated with 3 courses per student removed.  Each student was compared with the remaining $90\%$ (5337) to compute the top-$n$ most similar students. The results are shown in Table~\ref{cf-mae-table}.

The primary reason the estimates from CF are not much better than the estimate based on students' GPAs is that the dataset is too sparse to find similar users.  Upon analysis of similar users, we find that the similarity coefficients are very small ($O(10^-3)$).  We further validate this by noting that analysis shows the error of the student GPA estimates and the CF estimates are highly correlated.  Because the similarity coefficients determine the deviation of the estimate from the average, the CF estimates generated with small similarity coefficients are very similar to the student GPA estimates.

Recognizing that this user-based CF approach requires more data to be effective, we experimented with tuning various CF parameters in a non-rigorous manner with minor reductions in the MAE ($O(0.1)$).  We tried various values of $n$ from all students to $5$ at logarithmic intervals and found that smaller $n$ tended to reduce the MAE.  We also looked at the similarity coefficients: as defined in~\cite{breese}, the similarities are normalized before computing the weighted average.  By increasing the normalization coefficients, we further reduced the MAE.  We hypothesize that increasing the weights favors outliers and skews the estimate away from the average.  We also tried removing the influence of negatively correlated students as both Herlocker and Yao indicate that this improves error.  Though all these adjustments reduced the MAE, none significantly improved the estimates. 

For comparison with support vector machines, we ran tests on three random courses to compute the MAE for just those courses.  Here, for the chosen course $c$, we consider all students $S$ with $c \in C_{S_i} \forall{i}$.  We compare each student $S_i$ with students $S_j, j \neq i$ by computing the similarity $\textrm{sim}(C_{S_J}, C_{S_I} \\ c)$ and then using the weighted average as before to estimate $S_i$'s grade in $c$.  We used courses with over 50 students to run these tests.  The MAE for these tests did not differ significantly from the average student estimates.

\begin{table}[t]\scriptsize
\label{cf-mae-table}
\begin{center}

\begin{tabular}{lrrrr}
    
    &\multicolumn{1}{c}{\bf cosine sim}  &\multicolumn{1}{c}{\bf pearson's corr} &\multicolumn{1}{c}{\bf student's gpa} &\multicolumn{1}{c}{\bf mean course grade}
    \\ \hline \\
    MAE & 1.3896    &1.3704    &1.4231    &3.2140 \\
    $L_2$ error & 3.9439    &3.7505    &4.0451   &14.4421
\end{tabular}
\caption{MAE and $L_2$ error for estimates using collaborative filtering averaged over 25 test runs.  Each test estimated approximately 1,500 grades using 10,453 known course grades to compute the similarities using a base corpus of 5337 students.}

\end{center}
\end{table}

\section{Conclusion and Future Work}

Past course history can only go so far in predicting future course grades. A significant percentage of students in many of the courses had not taken any courses previously, which may partially explain the low accuracy of the baselines and the SVMs. In addition, there may be more promising features that characterize the current quarter if course schedule data were available, such as potential schedule and deadline conflicts between concurrent courses and student activities. Although the concurrent courses features was intended to help capture some of the effects of schedule conflict, it is flawed in that most courses do not have a fixed schedule and instead change term by term. Thus courses that have conflicted in the past may not conflict in the future.

It is also possible that the unexpected occurrences throughout the quarter may have greater influence on grades. For example, a student may just not have studied enough for a final exam, which heavily influences his or her grade. Individual personality factors may also play a part, where doing poorly on a similar course in the past may drive some students to work harder on a similar course in the future.

Improving the collaborative filtering results would most likely require additional data from the CourseRank system.  We would hope to find more similarities between students on which to base our estimations.  This may not be possible: though more grades will be added to the system each year, the relevancy of the older grades will diminish.

An alternative approach to address data sparsity might be to broaden our initial hypothesis.  We could broaden our hypothesis that similar students will perform similarly in courses.  Instead, given a student $s$ and course $c$, we could try to find students that have not only taken the same courses as $s$ an also taken $c$, but who have taken similar courses to both $c$ and courses similar to those taken by $s$.  Jun Wang, et. all have demonstrated a method similar to this that combines user-based and item-based collaborative filtering to overcome sparseness \cite{fusion}.  

Finally, it is possible that we see poor performance because the underlying assumption that similar students in courses from some set $C_a$ will receive similar grades in some other set $C_b$ may be false.  Factors such as concurrent course workload, interest in course, different professors, and others may instead be responsible for deviations in expected course grades.  Some of these external factors may be irrelevant as our analysis of SVMs indicates, but others may require collection of additional data before we can significantly improve course grade estimates.


%sample is the nips sample formatting guidlines, comment out when not needed anymore

%\section{Submission of papers to NIPS 2010}

NIPS requires electronic submissions.  The electronic submission site is  
\begin{center}
   http://papers.nips.cc
\end{center}

Please read carefully the
instructions below, and follow them faithfully.
\subsection{Style}

Papers to be submitted to NIPS 2010 must be prepared according to the
instructions presented here. Papers may be only up to eight pages long,
including figures. Since 2009 an additional ninth page \textit{containing only
cited references} is allowed. Papers that exceed nine pages will not be
reviewed, or in any other way considered for presentation at the conference.
%This is a strict upper bound. 

Please note that this year we have introduced automatic line number generation
into the style file (for \LaTeXe and Word versions). This is to help reviewers
refer to specific lines of the paper when they make their comments. Please do
NOT refer to these line numbers in your paper as they will be removed from the
style file for the final version of accepted papers.

The margins in 2010 are the same as since 2007, which allow for $\approx 15\%$
more words in the paper compared to earlier years. We are also again using 
double-blind reviewing. Both of these require the use of new style files.

Authors are required to use the NIPS \LaTeX{} style files obtainable at the
NIPS website as indicated below. Please make sure you use the current files and
not previous versions. Tweaking the style files may be grounds for rejection.

%% \subsection{Double-blind reviewing}

%% This year we are doing double-blind reviewing: the reviewers will not know 
%% who the authors of the paper are. For submission, the NIPS style file will 
%% automatically anonymize the author list at the beginning of the paper.

%% Please write your paper in such a way to preserve anonymity. Refer to
%% previous work by the author(s) in the third person, rather than first
%% person. Do not provide Web links to supporting material at an identifiable
%% web site.

%%\subsection{Electronic submission}
%%
%% \textbf{THE SUBMISSION DEADLINE IS JUNE 3, 2010. SUBMISSIONS MUST BE LOGGED BY
%% MIDNIGHT, JUNE 3, 2010, UNIVERSAL TIME}

%% You must enter your submission in the electronic submission form available at
%% the NIPS website listed above. You will be asked to enter paper title, name of
%% all authors, keyword(s), and data about the contact
%% author (name, full address, telephone, fax, and email). You will need to
%% upload an electronic (postscript or pdf) version of your paper.

%% You can upload more than one version of your paper, until the
%% submission deadline. We strongly recommended uploading your paper in
%% advance of the deadline, so you can avoid last-minute server congestion.
%%
%% Note that your submission is only valid if you get an e-mail
%% confirmation from the server. If you do not get such an e-mail, please
%% try uploading again. 


\subsection{Retrieval of style files}

The style files for NIPS and other conference information are available on the World Wide Web at
\begin{center}
   http://www.nips.cc/
\end{center}
%The author instructions for NIPS 2008 can also be found at 
%\begin{center}
%	 http://nips08.stanford.edu/nips08authors.html
%\end{center}
The file \verb+nips2010.pdf+ contains these 
instructions and illustrates the
various formatting requirements your NIPS paper must satisfy. \LaTeX{}
users can choose between two style files:
\verb+nips10submit_09.sty+ (to be used with \LaTeX{} version 2.09) and
\verb+nips10submit_e.sty+ (to be used with \LaTeX{}2e). The file
\verb+nips2010.tex+ may be used as a ``shell'' for writing your paper. All you
have to do is replace the author, title, abstract, and text of the paper with
your own. The file
\verb+nips2010.rtf+ is provided as a shell for MS Word users.

The formatting instructions contained in these style files are summarized in
sections \ref{gen_inst}, \ref{headings}, and \ref{others} below.

%% \subsection{Keywords for paper submission}
%% Your NIPS paper can be submitted with any of the following keywords (more than one keyword is possible for each paper):

%% \begin{verbatim}
%% Bioinformatics
%% Biological Vision
%% Brain Imaging and Brain Computer Interfacing
%% Clustering
%% Cognitive Science
%% Control and Reinforcement Learning
%% Dimensionality Reduction and Manifolds
%% Feature Selection
%% Gaussian Processes
%% Graphical Models
%% Hardware Technologies
%% Kernels
%% Learning Theory
%% Machine Vision
%% Margins and Boosting
%% Neural Networks
%% Neuroscience
%% Other Algorithms and Architectures
%% Other Applications
%% Semi-supervised Learning
%% Speech and Signal Processing
%% Text and Language Applications

%% \end{verbatim}

\section{General formatting instructions}
\label{gen_inst}

The text must be confined within a rectangle 5.5~inches (33~picas) wide and
9~inches (54~picas) long. The left margin is 1.5~inch (9~picas).
Use 10~point type with a vertical spacing of 11~points. Times New Roman is the
preferred typeface throughout. Paragraphs are separated by 1/2~line space,
with no indentation.

Paper title is 17~point, initial caps/lower case, bold, centered between
2~horizontal rules. Top rule is 4~points thick and bottom rule is 1~point
thick. Allow 1/4~inch space above and below title to rules. All pages should
start at 1~inch (6~picas) from the top of the page.

%The version of the paper submitted for review should have ``Anonymous Author(s)'' as the author of the paper.

For the final version, authors' names are
set in boldface, and each name is centered above the corresponding
address. The lead author's name is to be listed first (left-most), and
the co-authors' names (if different address) are set to follow. If
there is only one co-author, list both author and co-author side by side.

Please pay special attention to the instructions in section \ref{others}
regarding figures, tables, acknowledgments, and references.

\section{Headings: first level}
\label{headings}

First level headings are lower case (except for first word and proper nouns),
flush left, bold and in point size 12. One line space before the first level
heading and 1/2~line space after the first level heading.

\subsection{Headings: second level}

Second level headings are lower case (except for first word and proper nouns),
flush left, bold and in point size 10. One line space before the second level
heading and 1/2~line space after the second level heading.

\subsubsection{Headings: third level}

Third level headings are lower case (except for first word and proper nouns),
flush left, bold and in point size 10. One line space before the third level
heading and 1/2~line space after the third level heading.

\section{Citations, figures, tables, references}
\label{others}

These instructions apply to everyone, regardless of the formatter being used.

\subsection{Citations within the text}

Citations within the text should be numbered consecutively. The corresponding
number is to appear enclosed in square brackets, such as [1] or [2]-[5]. The
corresponding references are to be listed in the same order at the end of the
paper, in the \textbf{References} section. (Note: the standard
\textsc{Bib\TeX} style \texttt{unsrt} produces this.) As to the format of the
references themselves, any style is acceptable as long as it is used
consistently.

As submission is double blind, refer to your own published work in the 
third person. That is, use ``In the previous work of Jones et al.\ [4]'',
not ``In our previous work [4]''. If you cite your other papers that
are not widely available (e.g.\ a journal paper under review), use
anonymous author names in the citation, e.g.\ an author of the
form ``A.\ Anonymous''. 


\subsection{Footnotes}

Indicate footnotes with a number\footnote{Sample of the first footnote} in the
text. Place the footnotes at the bottom of the page on which they appear.
Precede the footnote with a horizontal rule of 2~inches
(12~picas).\footnote{Sample of the second footnote}

\subsection{Figures}

All artwork must be neat, clean, and legible. Lines should be dark
enough for purposes of reproduction; art work should not be
hand-drawn. The figure number and caption always appear after the
figure. Place one line space before the figure caption, and one line
space after the figure. The figure caption is lower case (except for
first word and proper nouns); figures are numbered consecutively.

Make sure the figure caption does not get separated from the figure.
Leave sufficient space to avoid splitting the figure and figure caption.

You may use color figures. 
% Apr 2010 -- sentence below is no longer true!
%However, please note that the archival version
%of the final proceedings are printed in greyscale. 
However, it is best for the
figure captions and the paper body to make sense if the paper is printed
either in black/white or in color.
\begin{figure}[h]
\begin{center}
%\framebox[4.0in]{$\;$}
\fbox{\rule[-.5cm]{0cm}{4cm} \rule[-.5cm]{4cm}{0cm}}
\end{center}
\caption{Sample figure caption.}
\end{figure}

\subsection{Tables}

All tables must be centered, neat, clean and legible. Do not use hand-drawn
tables. The table number and title always appear before the table. See
Table~\ref{sample-table}.

Place one line space before the table title, one line space after the table
title, and one line space after the table. The table title must be lower case
(except for first word and proper nouns); tables are numbered consecutively.

\begin{table}[t]
\caption{Sample table title}
\label{sample-table}
\begin{center}
\begin{tabular}{ll}
\multicolumn{1}{c}{\bf PART}  &\multicolumn{1}{c}{\bf DESCRIPTION}
\\ \hline \\
Dendrite         &Input terminal \\
Axon             &Output terminal \\
Soma             &Cell body (contains cell nucleus) \\
\end{tabular}
\end{center}
\end{table}

\section{Final instructions}
Do not change any aspects of the formatting parameters in the style files.
In particular, do not modify the width or length of the rectangle the text
should fit into, and do not change font sizes (except perhaps in the
\textbf{References} section; see below). Please note that pages should be
numbered.

\section{Preparing PostScript or PDF files}

Please prepare PostScript or PDF files with paper size ``US Letter'', and
not, for example, ``A4''. The -t
letter option on dvips will produce US Letter files.

Fonts were the main cause of problems in the past years. Your PDF file must
only contain Type 1 or Embedded TrueType fonts. Here are a few instructions
to achieve this.

\begin{itemize}

\item You can check which fonts a PDF files uses.  In Acrobat Reader,
select the menu Files$>$Document Properties$>$Fonts and select Show All Fonts. You can
also use the program \verb+pdffonts+ which comes with \verb+xpdf+ and is
available out-of-the-box on most Linux machines.

\item The IEEE has recommendations for generating PDF files whose fonts
are also acceptable for NIPS. Please see
http://www.emfield.org/icuwb2010/downloads/IEEE-PDF-SpecV32.pdf

\item LaTeX users:

\begin{itemize}

\item Consider directly generating PDF files using \verb+pdflatex+
(especially if you are a MiKTeX user). 
PDF figures must be substituted for EPS figures, however.

\item Otherwise, please generate your PostScript and PDF files with the following commands:
\begin{verbatim} 
dvips mypaper.dvi -t letter -Ppdf -G0 -o mypaper.ps
ps2pdf mypaper.ps mypaper.pdf
\end{verbatim}

Check that the PDF files only contains Type 1 fonts. 
%For the final version, please send us both the Postscript file and
%the PDF file. 

\item xfig "patterned" shapes are implemented with 
bitmap fonts.  Use "solid" shapes instead. 
\item The \verb+\bbold+ package almost always uses bitmap
fonts.  You can try the equivalent AMS Fonts with command
\begin{verbatim}
\usepackage[psamsfonts]{amssymb}
\end{verbatim}
 or use the following workaround for reals, natural and complex: 
\begin{verbatim}
\newcommand{\RR}{I\!\!R} %real numbers
\newcommand{\Nat}{I\!\!N} %natural numbers 
\newcommand{\CC}{I\!\!\!\!C} %complex numbers
\end{verbatim}

\item Sometimes the problematic fonts are used in figures
included in LaTeX files. The ghostscript program \verb+eps2eps+ is the simplest
way to clean such figures. For black and white figures, slightly better
results can be achieved with program \verb+potrace+.
\end{itemize}
\item MSWord and Windows users (via PDF file):
\begin{itemize}
\item Install the Microsoft Save as PDF Office 2007 Add-in from
http://www.microsoft.com/downloads/details.aspx?displaylang=en\&familyid=4d951911-3e7e-4ae6-b059-a2e79ed87041
\item Select ``Save or Publish to PDF'' from the Office or File menu
\end{itemize}
\item MSWord and Mac OS X users (via PDF file):
\begin{itemize}
\item From the print menu, click the PDF drop-down box, and select ``Save
as PDF...''
\end{itemize}
\item MSWord and Windows users (via PS file):
\begin{itemize}
\item To create a new printer
on your computer, install the AdobePS printer driver and the Adobe Distiller PPD file from
http://www.adobe.com/support/downloads/detail.jsp?ftpID=204 {\it Note:} You must reboot your PC after installing the
AdobePS driver for it to take effect.
\item To produce the ps file, select ``Print'' from the MS app, choose
the installed AdobePS printer, click on ``Properties'', click on ``Advanced.''
\item Set ``TrueType Font'' to be ``Download as Softfont''
\item Open the ``PostScript Options'' folder
\item Select ``PostScript Output Option'' to be ``Optimize for Portability''
\item Select ``TrueType Font Download Option'' to be ``Outline''
\item Select ``Send PostScript Error Handler'' to be ``No''
\item Click ``OK'' three times, print your file.
\item Now, use Adobe Acrobat Distiller or ps2pdf to create a PDF file from
the PS file. In Acrobat, check the option ``Embed all fonts'' if
applicable.
\end{itemize}

\end{itemize}
If your file contains Type 3 fonts or non embedded TrueType fonts, we will
ask you to fix it. 

\subsection{Margins in LaTeX}
 
Most of the margin problems come from figures positioned by hand using
\verb+\special+ or other commands. We suggest using the command
\verb+\includegraphics+
from the graphicx package. Always specify the figure width as a multiple of
the line width as in the example below using .eps graphics
\begin{verbatim}
   \usepackage[dvips]{graphicx} ... 
   \includegraphics[width=0.8\linewidth]{myfile.eps} 
\end{verbatim}
or % Apr 2009 addition
\begin{verbatim}
   \usepackage[pdftex]{graphicx} ... 
   \includegraphics[width=0.8\linewidth]{myfile.pdf} 
\end{verbatim}
for .pdf graphics. 
See section 4.4 in the graphics bundle documentation (http://www.ctan.org/tex-archive/macros/latex/required/graphics/grfguide.ps) 
 
A number of width problems arise when LaTeX cannot properly hyphenate a
line. Please give LaTeX hyphenation hints using the \verb+\-+ command.


\subsubsection*{Acknowledgments}

Use unnumbered third level headings for the acknowledgments. All
acknowledgments go at the end of the paper. Do not include 
acknowledgments in the anonymized submission, only in the 
final paper. 

\subsubsection*{References}

References follow the acknowledgments. Use unnumbered third level heading for
the references. Any choice of citation style is acceptable as long as you are
consistent. It is permissible to reduce the font size to `small' (9-point) 
when listing the references. {\bf Remember that this year you can use
a ninth page as long as it contains \emph{only} cited references.}

\small{
[1] Alexander, J.A. \& Mozer, M.C. (1995) Template-based algorithms
for connectionist rule extraction. In G. Tesauro, D. S. Touretzky
and T.K. Leen (eds.), {\it Advances in Neural Information Processing
Systems 7}, pp. 609-616. Cambridge, MA: MIT Press.

[2] Bower, J.M. \& Beeman, D. (1995) {\it The Book of GENESIS: Exploring
Realistic Neural Models with the GEneral NEural SImulation System.}
New York: TELOS/Springer-Verlag.

[3] Hasselmo, M.E., Schnell, E. \& Barkai, E. (1995) Dynamics of learning
and recall at excitatory recurrent synapses and cholinergic modulation
in rat hippocampal region CA3. {\it Journal of Neuroscience}
{\bf 15}(7):5249-5262.
}
\bibliographystyle{abbrv}
\bibliography{refs}
\end{document}
