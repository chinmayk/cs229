\subsection{Collaborative Filtering}

We consider another model for estimating grades based on the same methods used in the original recommendation system.  Here, we make the hypothesis that similar students will make similar grades.  Thus, for a student $s$ with course history $C_s$ and course $c \notin C_s$, we find students $s_i \in S$ with $|C_{s_i} \cup C_s| > 0$ and $c \in C_{s_i}$ (students who have taken some of the same courses as $s$ as well as course $c$).  Then $u$'s grade in $c$ can be estimated as a weighted average of the grades $g_{s_i}$.  

To determine the relative weights of users, we used cosine similarity and Pearson's correlation as the primary two measures.  Other measures of similarity such as Kendall, Spearman, and adjusted cosine similarity have been explored in~\cite{herlocker} and~\cite{yu}, but they tend to perform similarly or worse.  We expect our choice of similarity will have little impact on the predicted grades.

For two student (sparse) vectors $S_a$ and $S_b$ with each element corresponding to the student's grade in a course, we compute the similarity using:

\begin{center}
\begin{tabular}{cc}
$\textrm{similarity} = \frac{S_a \cdot S_b}{\|S_a\|\|S_b\|}$
&
$\textrm{Pearson's} = \frac{\textrm{cov}(S_a, S_b)}{\sigma_{S_a}\sigma_{S_b}} = \frac{\sum ^n _{i=1}(S_{a}^{(i)} - \overline{S}_a)(S_{b}^{(i)} - \overline{S}_b)}{\sqrt{\sum ^n _{i=1}(S_a^{(i)} - \overline{S}_a)^2} \sqrt{\sum ^n _{i=1}(S_b^{(i)} - \overline{S}_b)^2}}$.
\end{tabular}
\end{center}

Grade predictions for CF follow the multiclass SVM model with 1: A+, 2: A, 3: A-, $\ldots$, 13: F.  Other implementation details follow the formulas and algorithms outlined by~\cite{breese} and~\cite{cftoolkit}.

