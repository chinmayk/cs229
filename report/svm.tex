\subsection{SVM}
We trained an SVM to detect A versus non-A grades, where doing well in a course meant getting an A grade.  We also trained a multi-class SVM to predict grades across the whole letter grade spectrum from A+ to F. One SVM was trained per course.

All SVMs were  trained using a Gaussian kernel. The C and gamma parameters for the kernel were chosen via a rough grid-based search on C={128 64 32 8 4 2 1} and gamma={1/numFeatures 2/numFeatures 4/numFeatures 8/numFeatures} to maximize the 10-fold cross validation accuracy. The multi-class SVM was implemented using multiple SVMs that compared grades higher than a certain threshold against the grades lower than the threshold. That is, there is one SVM that predicts each of  A+ vs other grades, A or above vs. below A, A- or above vs. below A-, and so on for each unique grade appearing in the course. The final predicted grade is then decided by a majority vote among all the individual SVMs.


Given the CourseRank transcript data, we chose six features that we hypothesized may affect course grades-- the students' previous course grades, recent GPA by department (last three quarters), major, concurrent courses, planned weekly workload, and the number of courses previously taken.

These features provide some information on the students' experience with related courses and subjects at Stanford, whether the course is a requirement or an elective for their major, and how much time they have to spend on the course. The previous course grades, recent GPA by department, student major, and concurrent courses were represented by vectors where each index corresponded to one course, department, or major.




